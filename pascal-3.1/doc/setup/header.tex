%%% Spracheinstellung & Codierung
  %% Pakete
    % Sprache & Codierung
      \usepackage[utf8x]{inputenc}      % utf8 Codierung der tex-Dateien
      \usepackage{ucs}        % Unicode in Überschriften/Referenzen/Variablen
      \usepackage[ngerman,english]{babel}   % Deutsche und englische Spracheinstellung
      \usepackage{hyphsubst}        % Trennung an Wortfugen
      \usepackage[T1]{fontenc}      % T1 Computer-Modern (bessere Trennung!)
    % Zitate
      \usepackage{cite}
      \usepackage{bibgerm}
%       \usepackage[square]{natbib}
%       \usepackage[numbers,round]{natbib}
%         \makeatletter
%         \def\newblock{\beamer@newblock}
%         \makeatother
%           \bibliographystyle{abbrdinat}
%           \bibliographystyle{abbrvnat}
%           \bibliographystyle{unsrtnat}
%           \bibliographystyle{plain}
%           \bibliographystyle{alphadin}
%           \bibliographystyle{apalike}
  %% Einstellungen
    \hyphenation{%
      % Bei Wörtern mit Bindestrichen im Text "" hinter Bindestrich un dann \textminus mit \usepackage{textcomp}, etc.
      Dif-fe-ren-ti-a-ti-ons-ma-t-ri-zen
      Ge-schwin-dig-keits
      Spek-tral
      Ele-men-te
      Pha-sen-feld
      Kern-thema
    }

%%% Angaben zum Dokument, Autor, Datum, etc.
  \newcommand{\MyDocTitleOne}{Documentation of the development for foam-3.1 user extensions}
  \newcommand{\MyDocTitleTwo}{}
  \newcommand{\MyDocTitleTrd}{}
  \newcommand{\MyDocType}{Report}
  \newcommand{\MyDocYear}{2015}
  \newcommand{\MyDocStart}{17.3.2015}
  \newcommand{\MyDocEnd}{?}
  \newcommand{\MyDocLastchange}{\today}

  \newcommand{\MyKeyWords}{}

  \newcommand{\MyFirstName}{Pascal}
  \newcommand{\MyLastName}{Beckstein}
  \newcommand{\MyName}{\MyFirstName \ \MyLastName}
  \newcommand{\MyBirthdayShort}{08.10.1984}
  \newcommand{\MyBirthdayLong}{08. Oktober 1984}
  \newcommand{\MyBirthplace}{Nürnberg}

  \newcommand{\MySupervisor}{Dr. V. Galindo}
  \newcommand{\MyProfessorOne}{?}
  \newcommand{\MyProfessorTwo}{?}

  \newcommand{\MyReviewerOne}{?}
  \newcommand{\MyReviewerTwo}{?}

  \title{\Mytitle}
  \author{\MyName}
  \date{\MyLastchange}

%%% KOMA Optionen (Satzspiegel wird später neu berechnet)
  \KOMAoptions{draft=true}        % Entwurfsstadium (false)
  \KOMAoptions{paper=a4}          % Papierformat (a4)
  \KOMAoptions{pagesize=auto}       % Ausgabetreiber/Korrekte Ausgabe der Papiergröße
  \KOMAoptions{twoside=true}        % Zweiseitiges Layout (false)
  \KOMAoptions{open=any}          % Kapitelanfänge bei zweiseitigem Druck (any)
                % \KOMAoptions{open=left,cleardoublepage=plain}% Kapitelanfänge bei zweiseitigem Druck (any)
  \KOMAoptions{BCOR=15mm}         % Bindungskorrektur (0pmm)
  \KOMAoptions{DIV=15}          % Seitenteilung
  \KOMAoptions{fontsize=11pt}       % Schriftgröße
%   \KOMAoptions{abstract=false}        % Überschrift der Zusammenfassung (false): NICHT BEI scrbook
  \KOMAoptions{parskip=half-}       % Absatzauszeichnung (false)
  \KOMAoptions{headinclude=true}        % Kopfzeile zum Textkörper? (false)
  \KOMAoptions{footinclude=true}        % Fußzeile zum Textkörper? (false)
%   \KOMAoptions{headings=normal}       % Größe der Überschriften
  \KOMAoptions{headlines=2.1}       % Größe der Kopfzeile (1.25)
  \KOMAoptions{listof=totoc}        % Abbildungsverzeichnisse - TOC & Nummerierung
  \KOMAoptions{bibliography=totoc}      % Literaturverzeichnis - TOC & Nummerierung
  \KOMAoptions{numbers=noenddot}        % Kein Punkt am Ende

%%% Allgemeine Pakete
  %% Pakete
    \usepackage{scrhack}        % Remove warning for listings and float
%     \usepackage{gnuplottex}       % Gnuplot in latex
%     \usepackage{pgfplots}       % Diagramme und Plots mit tikzpicture
%     \usepackage{filecontents}     % Dateiinhalte schreiben
    \usepackage{listings}       % Quellcode
    \usepackage{graphicx}       % Grafiken einbinden
      \graphicspath{
        {pictures/}%
        {logos/}%
      }
    \usepackage{float}        % H option for floats and new floats
    \usepackage[verbose]{placeins} % \FloatBarrier
    \usepackage{subfig}       % Mehrere Figuren in einer Umgebung
%     \usepackage{psfrag}       % Schlüsselwörter in Postscript Dateien ersetzen und andere Tools
    \usepackage{paralist}       % Erweiterte Listenumgebung
    \usepackage{color,framed}     % farbiger Text und Hintergrund
    \usepackage[usenames,dvipsnames,svgnames,table]{xcolor}
    \usepackage{colortbl}       % farbige Einstellungen für Tabellen
%     \usepackage{verbatim}       % Textumgebung
    \usepackage{fancyvrb}       % Bessere Textumgebung
%     \usepackage{rotating}         % Objekte drehen (\begin{sidewaystable}..., \begin{sidewaysfigure}...)
    \usepackage[scaled=1]{helvet}     % Helvetica Schrift (setzt \sfdefault neu!)
    \usepackage{scrpage2}       % Kopf und Fußzeilen
%     \usepackage{caption}        % Bessere Abbildungsbeschriftungen, wird bereits von float geladen
    \usepackage{afterpage}        % \afterpage{\clearpage}
    \usepackage[warn]{textcomp}     % besondere Symbole
    \usepackage{nameref}        % Referenzen zu Namen
  %% Einstellungen
    % Worttrennung und Satz [ badness = [-10000 ... 10000] ! ]
      \raggedbottom         % Text muss auf Seitenende NICHT abschließen (unsauber!)
%       \flushbottom          % Text muss auf Seitenende abschließen
      \pretolerance=100       % Zulässige 'badness' OHNE Worttrennung (100)
      \tolerance=200          % Zulässige 'badness' MIT Worttrennung (200)
      \hfuzz=0.1pt          % Grenze für horiz. 'overfull hbox'-Meldung (0.1pt)
      \vfuzz=0.1pt          % Grenze für vert. 'overfull vbox'-Meldung (0.1pt)
      \hbadness=1000          % Grenze für horiz. 'badbox'-Meldung (1000)
      \vbadness=1242          % Grenze für vert. 'badbox'-Meldung (1000)
      \emergencystretch=1pt       % Zusätzlicher Platz für Streckung (0pt)
      \linepenalty=10         % Strafe: Zeilenanzahl (10)
      \hyphenpenalty=50       % Strafe: Trennungen (50)
      \exhyphenpenalty=50       % Strafe: Expl. Trennungen (50)
      \binoppenalty=700       % Strafe: TEXTMATH-Trennung nach binärem Op. (700)
      \relpenalty=500         % Strafe: TEXTMATH-Trennung nach Relation (500)
      \clubpenalty=150        % Strafe: Erste Zeile eines Absatzes bleibt allein auf vorh. Seite (150)
      \widowpenalty=150       % Strafe: Letzte Zeile v. Absatz wird alleine auf nächste Seite umgebr. (150)
      \doublehyphendemerits=10000     % Strafe: Trennung in zwei hintereinander kommenden Zeilen (10000)
      \finalhyphendemerits=5000     % Strafe: Trennung in der vorletzten Zeile eines Absatzes (5000)
      \adjdemerits=10000        % Strafe: Zwei visuell inkompatible Zeilen hintereinander (10000)
      \interdisplaylinepenalty=100      % Strafe: Umbruch von '\displaylines-Formelfolge' über Seiten (100)
      \interfootnotelinepenalty=100     % Strafe: Umbruch einer Fußnote über Seitengrenzen hinweg (100)
      \interlinepenalty=0       % Strafe: Umbruch eines Absatzes über Seite (0)
      \predisplaypenalty=10000      % Strafe: Seitenumbruch vor Formel im display-style (10000)
      \postdisplaypenalty=0       % Strafe: Seitenumbruch nach Formel im display-style (0)
    % Gliederungstiefe im Text/TOC
      \setcounter{secnumdepth}{3}
      \setcounter{tocdepth}{3}      % Gliederungstiefe (2:subsection)
    % Schriftarten
      \renewcommand{\familydefault}{\sfdefault}       % Fontfamilie
      \fontfamily{phv}\selectfont
%       \setkomafont{pageheadfoot}{\sc\sffamily\normalcolor\normalsize} % Kopf-/Fußzeile
%       \setkomafont{caption}{\sffamily\normalcolor\normalsize}   % Text zum Label einer Figure
%       \setkomafont{captionlabel}{\bf\sffamily\normalcolor\normalsize} % Label einer Figure
    % Kapitelanfänge
      \renewcommand{\chapterheadstartvskip}{\vspace*{-3.0\topskip}}
    % Kopf- und Fußzeilen
      \pagestyle{scrheadings}         % Seitenaussehen
      \ihead[]{}            % Eintrag oben innen
      \chead[]{}            % Eintrag oben mitte
      \ohead[]{}            % Eintrag oben außen
      \ohead[]{\leftmark}         % Eintrag oben außen
      \ohead[]{\headmark}         % Eintrag oben außen
      \setheadsepline{0.5pt}          % Linie unterhalb der Kopfzeile
      \ifoot[]{}            % Eintrag unten innen
      \cfoot[]{}            % Eintrag unten mitte
      \ofoot[\pagemark]{\pagemark}        % Eintrag unten außen
      \setfootsepline{0.5pt}          % Linie oberhalb der Fußzeile
    % Captions allgemein
      \captionsetup{format=plain,singlelinecheck=true}  % Formatangaben der Beschriftungen
    % Quellcodetype
%       \lstset{language=C++}
    % Floats
      %   General parameters, for ALL pages:
        \renewcommand{\topfraction}{0.9}  % max fraction of floats at top
        \renewcommand{\bottomfraction}{0.8} % max fraction of floats at bottom
      %   Parameters for TEXT pages (not float pages):
        \setcounter{topnumber}{2}
        \setcounter{bottomnumber}{2}
        \setcounter{totalnumber}{4}     % 2 may work better
        \setcounter{dbltopnumber}{2}    % for 2-column pages
        \renewcommand{\dbltopfraction}{0.9} % fit big float above 2-col. text
        \renewcommand{\textfraction}{0.07}  % allow minimal text w. figs
      %   Parameters for FLOAT pages (not text pages):
        \renewcommand{\floatpagefraction}{0.7}  % require fuller float pages
        \renewcommand{\dblfloatpagefraction}{0.7} % require fuller float pages (MUST be less than topfraction)

%%% Festlegen des Seitenlayouts
  %% Satzspielgel von KOMA nach geänderten KOMAoptions neu berechnen (!!! Diesen Abschnitt NIEMALS vor die Formatierung packen!!!)
%     \KOMAoptions{DIV=last}
  %% Geometrie festlegen mit geometry-Paket - Überschreibt den KOMA Satzspielgel:
    \usepackage[top=2.3cm,hcentering,textwidth=155mm,textheight=244mm]{geometry}
    \setlength{\headheight}{1.5\baselineskip}
  %% Zeilenabstand
%     \usepackage[onehalfspacing]{setspace}   % Zeilenabstand in ECHTEM 1,5 Zeilenabstand
    \usepackage{setspace}\setstretch{1.4}   % Zeilenabstand in WORD-1,5-Zeilenabstand

%%% Hyperref & pdf (hyperref > \texorpdfstring{$...$}{ ... })
  %% Paket und Optionen
    \usepackage[%
          author=\MyName,
          icolor=Ivory
          ]{pdfcomment}     % pdfcomments (\pdfcomment[icon=Comment,color=YellowGreen]{},
                  %              \pdfcomment[icon=Note,color=Khaki]{},
                  %              \pdfmargincomment[icon=Help,color=Tomato]{})
    \usepackage{hypcap}         % Hyperlink captions
    \hypersetup{%
        % General
          bookmarksdepth=4    % Bookmark Tiefe
          plainpages=false,   % Arabische Bezeichener
          hypertexnames=true,   % Sinnvolle Namen
        % Extensions
%           hyperfigures=true,    % Links für Figuren
%           pagebackref=false,    % Referenz von Quellen zurück auf Seiten
%           hyperfootnotes=true,    % Fußnoten verlinken
        % Farben
          colorlinks,     % Farbige Links
          allcolors=black,    % Alle Linkfarben
        % Bookmarks
          bookmarksopen=true,   % Offene Baumstruktur in den Bookmarks
          bookmarksnumbered=true,   % Nummerierte Bookmarks
          bookmarksopenlevel={2},   % Bis zu welcher Tiefe sollen die Bookmarks geöffnet werden?
        % PDF Optionen
          pdftoolbar=true,    % PDF Toolbar
          pdfmenubar=true,    % PDF Menubar
          pdfpagelayout={TwoPageRight}, % PDF Layout
        % Titel & Autor & Thema
          pdfauthor={\MyName},
          pdftitle={\MyDocTitleOne \ \MyDocTitleTwo \ \MyDocTitleTrd},
          pdfsubject={\MyDocType},
          pdfkeywords={\MyKeyWords}
    }
%%% Tabellen
  %% Pakete
    \usepackage{booktabs}   % Bessere Tabellen
    \usepackage{hhline}   % Bessere horizontale Linien
    \usepackage{tabularx}   % Tabellen mit fester Gesamtbreite, weitere Optionen
    \usepackage{longtable}    % Mehrseitige Tabellen
    \usepackage{multirow}   % Mehrere Reihen/Spalten zusammenfassen (multirow/multicolumn)
  %% Einstellungen
    % Zeilenabstand
      \renewcommand{\arraystretch}{1.4}         % Array-Einträge horizontal weiter auseinander (1)
    % Textausrichtung bei fester Spaltenbreite (braucht tabularx)
      \newcolumntype{L}[1]{>{\raggedright\arraybackslash}p{#1}}   % Breitenangabe & linksbündig
      \newcolumntype{C}[1]{>{\centering\arraybackslash}p{#1}}   % Breitenangabe & zentriert
      \newcolumntype{R}[1]{>{\raggedleft\arraybackslash}p{#1}}    % Breitenangabe & rechtsbündig
    % Andere Textausrichtung innerhalb einer Spalte mit vordefinierter Ausrichtung
      \newcommand{\ltab}{\raggedright\arraybackslash}     % Tabellenabschnitt linksbündig
      \newcommand{\ctab}{\centering\arraybackslash}       % Tabellenabschnitt zentriert
      \newcommand{\rtab}{\raggedleft\arraybackslash}        % Tabellenabschnitt rechtsbündig
    % Caption setup
      \captionsetup[table]{   % Format Beschriftung von Tabellen
        format=hang,
        justification=justified,
        singlelinecheck=true,
        aboveskip=0pt,
        belowskip=9pt
      }
      \captionsetup[longtable]{ % Format Beschriftung von langen Tabellen
        format=hang,
        justification=justified,
        singlelinecheck=true,
        aboveskip=0pt,
        belowskip=9pt
      }

%%% Figuren, Abbildungen
  %% Pakete
    \usepackage{wrapfig}      % Umflossene floats
  %% Einstellungen
    % Caption setup
      \captionsetup[figure]{    % Format der Beschriftung von Abbildungen
        format=hang,
        justification=justified,
        singlelinecheck=true,
        aboveskip=9pt,
        belowskip=0pt
      }

%%% Mathematik
  %% Pakete
    \usepackage{amsmath,amssymb,amsthm,amsfonts} % Matheumgebunstools
    \usepackage{mathtools}                       % multlined Umgebung
    \usepackage{units}                           % Einheiten
    \usepackage{cancel}                          % Terme durchstreichen
    \usepackage{upgreek}                         % Upside greek letters
    \usepackage{esint}                           % Weitere Integralzeichen
    \usepackage{gensymb}                         % Weitere Symbole
  %% Einstellungen

%%% Farben
  \definecolor{tublue}{rgb}{0.04,0.16,0.32}
  \definecolor{greycust}{rgb}{0.50,0.50,0.50}
  \definecolor{tubluelight}{rgb}{0.46,0.50,0.62}
  \definecolor{darkgreen}{rgb}{0.00, 0.48, 0.28}
  \definecolor{lightred}{rgb}{1.00, 0.90, 0.90}
  \definecolor{darkred}{rgb}{0.80, 0.00, 0.00}
  \definecolor{altcol}{rgb}{0.8,0.0,0.0}
  \definecolor{gray}{rgb}{0.9,0.9,0.9}
  \definecolor{shadecolor}{rgb}{0.9,0.9,0.9}

%%% Neue Kommandos für spezielle Symbole
  %% Besondere Operatoren
    % Signum
      \newcommand{\sgn}{\operatorname{sgn}}
      \newcommand{\sign}{\operatorname{sign}}
    % Soll-Gleichheit
      \newcommand{\forceeq}{\mathop{\stackrel{!}{=}}}
    % Beschriftete Gleichheit
      \newcommand{\texteq}[1]{\mathop{\stackrel{\text{#1}}{=}}}
    % Def-Gleichheit
      \newcommand{\defeql}{\mathop{:=}}
      \newcommand{\defeqr}{\mathop{=:}}
    % Betrag / Norm
      \newcommand{\abs}[1]{\left| #1 \right|}
      \newcommand{\norm}[1]{\lVert #1 \rVert}

  %% Komplexe Zahlen
      \newcommand{\re}[1]{\mathrm{Re}\left[ #1 \right]}
      \newcommand{\im}[1]{\mathrm{Im}\left[ #1 \right]}
      \newcommand{\ad}[1]{#1^{*}}

  %% Vektoren und Matrizen
    % Vektor
%       \renewcommand{\vec}[1]{\boldsymbol{#1}}
      \renewcommand{\vec}[1]{\bvec{#1}}
    % Tensor
      \newcommand{\ten}[1]{\vec{#1}}
    % Gradient
      \newcommand{\grad}[1]{\nabla #1}
    % Vektorgradient
      \newcommand{\vgrad}[1]{\grad{\vec{#1}}}
    % Laplace
      \newcommand{\laplace}[1]{\nabla^2 #1}
    % Divergenz
      \renewcommand{\div}[1]{\nabla \cdot #1}
    % Vektordivergenz
      \newcommand{\vdiv}[1]{\div{\vec{#1}}}
    % Konvektion (konservativ)
      \newcommand{\conv}[1]{\div{\vec{#1}\vec{#1}}}
      \newcommand{\convcons}[1]{\conv{#1}}
    % Konvektion (konvektiv)
      \newcommand{\convconv}[1]{\left( \vec{#1} \cdot \nabla \right) \vec{#1} }
    % Konvektion (rotational)
      \newcommand{\convrot}[1]{\frac{1}{2} \grad{\vec{#1}^2} - \vec{#1} \times \left( \rot{\vec{#1}} \right)}
    % Konvektion (schief-symmetrisch)
      \newcommand{\convskew}[1]{\frac{1}{2}\left[ \convcons{#1} + \convconv{#1} \right]}
    % Rotation
      \newcommand{\rot}[1]{\nabla \times #1}
    % Indexdarstellung
      \newcommand{\vecindex}[3]{\left\lbrace #1 \right\rbrace_{#2 #3}}
    % Transposition
      \newcommand{\transpose}[1]{\left\lbrace #1 \right\rbrace^T}

  %% Ableitungen
    % 1. Ordnung
      \newcommand{\fd}[2]{\frac{\mathrm{d} #1}{\mathrm{d} #2}}
    % 2. Ordnung
      \newcommand{\fdd}[2]{\frac{\mathrm{d}^2 #1}{\mathrm{d} #2^2}}
    % 3. Ordnung
      \newcommand{\fddd}[2]{\frac{\mathrm{d}^3 #1}{\mathrm{d} #2^3}}
    % 4. Ordnung
      \newcommand{\fdddd}[2]{\frac{\mathrm{d}^4 #1}{\mathrm{d} #2^4}}

  %% Partielle Ableitungen
    % 1. Ordnung
      \newcommand{\fpd}[2]{\frac{\partial #1}{\partial #2}}
      \newcommand{\spd}[2]{\partial_#2 #1}
    % 2. Ordnung
      \newcommand{\fpdd}[2]{\frac{\partial^2 #1}{\partial #2^2}}
      \newcommand{\spdd}[2]{\partial_{#2 #2} #1}
    % 3. Ordnung
      \newcommand{\fpddd}[2]{\frac{\partial^3 #1}{\partial #2^3}}
      \newcommand{\spddd}[2]{\partial_{#2 #2 #2} #1}
    % 4. Ordnung
      \newcommand{\fpdddd}[2]{\frac{\partial^4 #1}{\partial #2^4}}
      \newcommand{\spdddd}[2]{\partial_{#2 #2 #2 #2} #1}

  %% Materielle Ableitung
    \newcommand{\fmd}[2]{\frac{\mathrm{D} #1}{\mathrm{D} #2}}

  %% 1. Variationsableitung
    \newcommand{\vard}[2]{\frac{\mathrm{\delta} #1}{\mathrm{\delta} #2}}

  %% 1. Eulerableitung
    \newcommand{\varde}[2]{\frac{\hat{\partial} #1}{\hat{\partial} #2}}

  %% Integrale
    % Integrationsvariable
      \newcommand{\intvar}[1]{\; \mathrm{d}#1}
      \newcommand{\vintvar}[1]{\cdot \mathrm{d}\vec{#1}}
    % Linienintegral
      \newcommand{\integral}[4]{\int_{#1}^{#2} #3 \intvar{#4}}
      \newcommand{\vintegral}[4]{\int_{#1}^{#2} #3 \vintvar{#4}}
    % Flächenintegrale
      \newcommand{\iintegral}[4]{\iint_{#1}^{#2} #3 \intvar{#4}}
      \newcommand{\viintegral}[4]{\iint_{#1}^{#2} #3 \vintvar{#4}}
      \newcommand{\ointegral}[4]{\oint_{#1}^{#2} #3 \intvar{#4}}
      \newcommand{\ovintegral}[4]{\iint_{#1}^{#2} #3 \vintvar{#4}}
    % Volumenintegral
      \newcommand{\iiintegral}[4]{\iiint_{#1}^{#2} #3 \intvar{#4}}

  %% Mittelung
      \newcommand{\avg}[2]{\left<#1\right>_{#2}}

  %% Ränder- und Bedingungen
      \newcommand{\bc}[2]{\left. #1 \right|_{#2}}

  %% Fehlerordnung
      \newcommand{\ord}[1]{\mathcal{O}\left(#1\right)}

  %% Zeitdiskretisierung
    % Markierung
      \newcommand{\tdis}[2]{#1^{\, \langle #2 \rangle}}
    % Zeitextrapolationen
      \newcommand{\told}{\textcolor{DarkGreen}{\bigstar}}
      \newcommand{\texp}{\textcolor{DarkBlue}{\bullet}}
      \newcommand{\timp}{\textcolor{DarkRed}{\blacklozenge}}

  %% Phasen
    % Markierungen
      \newcommand{\phaseone}[1]{#1_{\mbox{\tiny $I$}}}
      \newcommand{\phasetwo}[1]{#1_{\mbox{\tiny $II$}}}
      \newcommand{\phaseint}[1]{#1_{\mbox{\tiny $I/II$}}}

  %% Expontielle Darstellungen
    % bt
      \newcommand{\bt}[1]{\times 10^{#1}}

%%% Symbole aus anderen Schriftarten

  %% Geboren/Gestorben
    \let\born\textborn
    \let\died\textdied
    \renewcommand{\textborn}{{\fontfamily{lmr}\selectfont\born}}
    \renewcommand{\textdied}{{\fontfamily{lmr}\selectfont\died}}

%%% Theoreme
  \theoremstyle{definition}
  \newtheorem{definition}{Definition}[subsection]
  \newtheorem{theorem}[definition]{Satz}
  \newtheorem{lem}[definition]{Lemma}

  \newtheoremstyle{myboxstyle}
    {3pt}         % Space above
    {3pt}         % Space below
    {\itshape}        % Body font: original {\normalfont}
    {}          % Indent amount (empty = no indent,\parindent = para indent)
    {\normalfont\bfseries}      % Thm head font original {\normalfont\bfseries}
    {:}         % Punctuation after thm head original :
    { }         % Space after thm head: " " = normal interword
    {}          % Theorem head spec (can be left empty, meaning ‘normal’)

  \theoremstyle{myboxstyle}
  \newtheorem*{remark}{Hinweis}%[chapter]
  \newtheorem*{question}{Frage}%[chapter]
  \newtheorem*{todo}{Aufgabe}%[chapter]

%%% Umgebungen
  \newenvironment{remarkbox}{\definecolor{shadecolor}{rgb}{0.85,1.0,0.85}\begin{shaded}\begin{remark}}{\end{remark}\end{shaded}}
  \newenvironment{questionbox}{\definecolor{shadecolor}{rgb}{1.0,0.85,0.85}\begin{shaded}\begin{question}}{\end{question}\end{shaded}}
  \newenvironment{todobox}{\definecolor{shadecolor}{rgb}{1.0,1.0,0.7}\begin{shaded}\begin{todo}}{\end{todo}\end{shaded}}

%%% Neue floats
  % Einfacher Quellcode
%     \floatstyle{ruled}
%     \newfloat{code}{thp}{lop1}
%     \floatname{code}{Quellcode-Auszug}
%     \captionsetup[code]{format=hang,justification=justified,singlelinecheck=true,aboveskip=9pt,belowskip=0pt}
  % Beispiele
    \floatstyle{ruled}
    \newfloat{example}{thp}{lop2}
    \floatname{example}{Beispiel}
    \captionsetup[example]{format=hang,justification=justified,singlelinecheck=true}